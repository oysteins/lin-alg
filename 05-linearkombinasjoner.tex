\input{kapittel}

\kapittel{5}{Lineærkombinasjoner}
\label{ch:linearkombinasjoner}

I forrige kapittel begynte vi å endre på hvordan vi tenker på lineære
likningssystemer.  Nå kan vi tenke på et lineært likningssystem som én
enkelt likning
\[
\a_1 x_1 + \a_2 x_2 + \cdots + \a_n x_n = \b
\]
der hver av $\a$-ene er en vektor i $\R^m$, og $\b$ er en vektor
i~$\R^n$.  Det å finne ut om systemet har en løsning blir nå det samme
som å finne ut om det er mulig å «komme frem til» $\b$ ved å bruke
$\a$-ene.
I dette kapitlet utforsker vi denne ideen litt videre.

% TODO mer her?


\section*{Hvor kan jeg komme med disse vektorene?}

Se på disse tre vektorene i~$\R^2$:
\begin{center}
\begin{tikzpicture}[scale=.6]
\draw[->] (-2.5,0) -- (4.5,0);
\draw[->] (0,-1.5) -- (0,3.5);
\foreach \x in {-2,...,-1,1,2,...,4}
\draw (\x,.1) -- (\x,-.1);% node[anchor=north] {$\x$};
\foreach \y in {-1,1,2,...,3}
\draw (.1,\y) -- (-.1,\y);% node[anchor=east] {$\y$};
\draw[->] (0,0) -- (4,2) node[anchor=west] {$\u = \vv{4}{2}$};
\draw[->] (0,0) -- (-2,-1) node[anchor=north] {$\v = \vv{-2}{-1}$};
\filldraw (3,-1) circle [radius=2pt] node[anchor=west] {$\b = \vv{3}{-1}$};
% TODO b som pil?
\end{tikzpicture}
\end{center}
% TODO kalle dem a1, a2, b eller u, v, w?
Kan vi komme til $\b$ ved å kombinere $\u$ og~$\v$?

Vi ser ganske raskt at hvis vi prøver å skalere $\u$ og~$\v$ med hvert
sitt tall og legge dem sammen -- for eksempel
\[
2\u + 3\v
= 2 \vv{4}{2} + 3 \vv{-2}{-1}
= \vv{2}{1}
\]
eller
\[
\u + \frac{5}{2} \cdot \v
= \vv{4}{2} + \frac{5}{2} \cdot \vv{-2}{-1}
= \vv{-1}{-1/2}
\]
-- så havner vi bare forskjellige steder på denne rette linjen:
\begin{center}
\begin{tikzpicture}[scale=.6]
\draw[->] (-4.5,0) -- (4.5,0);
\draw[->] (0,-2.5) -- (0,2.5);
\foreach \x in {-4,...,-1,1,2,...,4}
\draw (\x,.1) -- (\x,-.1);
\foreach \y in {-2,-1,1,2}
\draw (.1,\y) -- (-.1,\y);
\draw (-4.4,-2.2) -- (4.4,2.2);
\end{tikzpicture}
\end{center}
Vi klarer aldri å komme oss bort fra denne linjen.  Dermed klarer vi
heller ikke å komme oss til~$\b$.

Spørsmålet om hvorvidt vi kan komme oss til~$\b$ ved å kombinere $\u$
og~$\v$ kan også formuleres som en likning:
\[
\u x + \v y = \b
\]
Vi kan prøve å løse den med gausseliminasjon:
\[
\begin{amatrix}{2}
4 & -2 & 3 \\
2 & -1 & -1
\end{amatrix}
\roweq
\begin{amatrix}{2}
4 & -2 & 3 \\
0 & 0 & -5/2
\end{amatrix}
\]
Her har vi fått totalmatrisen på trappeform, og vi har et pivotelement
til høyre for streken (den nederste raden svarer til likningen
$0 = -5/2$).  Det betyr at likningen -- som forventet -- ikke har noen
løsning.

\smallskip%
Det var et eksempel på at vi ikke kommer oss dit vi vil ved å
kombinere vektorene vi har.  Så kan vi forsøke å endre scenariet vårt
en smule ved å vri littegrann på den ene vektoren, og se hva som skjer:
\begin{center}
\begin{tikzpicture}[scale=.6]
\draw[->] (-2.5,0) -- (4.5,0);
\draw[->] (0,-1.5) -- (0,3.5);
\foreach \x in {-2,...,-1,1,2,...,4}
\draw (\x,.1) -- (\x,-.1);% node[anchor=north] {$\x$};
\foreach \y in {-1,1,2,...,3}
\draw (.1,\y) -- (-.1,\y);% node[anchor=east] {$\y$};
\draw[->] (0,0) -- (4,2) node[anchor=west] {$\u = \vv{4}{2}$};
\draw[->] (0,0) -- (-2,-.8) node[anchor=north] {$\w = \vv{-2}{-4/5}$};
\filldraw (3,-1) circle [radius=2pt] node[anchor=west] {$\b = \vv{3}{-1}$};
\end{tikzpicture}
\end{center}
Her har vi den samme vektoren~$\u$ som før, og så har vi en ny
vektor~$\w$ som er nesten, men ikke helt, lik~$\v$.  Kan vi komme
oss til~$\b$ ved å kombinere $\u$ og~$\w$?

Det er i hvert fall klart at vi ikke lenger er begrenset til å bevege
oss langs én rett linje.  Ved å skalere $\u$ med forskjellige tall kan
vi fremdeles komme oss hvor som helst på den samme linjen som før, og
ved å legge til en skalering av~$\w$ kommer vi oss ut av den linjen og
til andre steder i planet.

Nå kan vi gjette at vi kan klare å komme oss til et hvilket som helst
sted i planet ved å kombinere $\u$ og~$\w$.  Skal vi sjekke om vi
kommer oss til~$\b$?  Vi bør kunne få det til, men det er ikke akkurat
åpenbart hvilke tall vi må skalere $\u$ og~$\w$ med for å komme
til~$\b$.

La oss formulere problemet som en likning:
\[
\u x + \w y = \b
\]
Eller, om du vil, som et likningssystem:
\[
\systeme{
  4x - 2y = 3,
  2x - \frac{4}{5}y = -1
  }
\]
Nå kan vi ty til vår gamle venn gausseliminasjon:
\begin{align*}
\begin{amatrix}{2}
4 & -2 & 3 \\
2 & -\frac{4}{5} & -1
\end{amatrix}
&\roweq
\begin{amatrix}{2}
4 & -2 & 3 \\
0 & \frac{1}{5} & -\frac{5}{2}
\end{amatrix}
\\
&\roweq
\begin{amatrix}{2}
4 & -2 & 3 \\
0 & 1 & -\frac{25}{2}
\end{amatrix}
\\
% &\roweq
% \begin{amatrix}{2}
% 4 & 0 & -22 \\
% 0 & 1 & -\frac{25}{2}
% \end{amatrix}
% \\
&
\roweq
\begin{amatrix}{2}
1 & 0 & -11/2 \\
0 & 1 & -25/2
\end{amatrix}
\end{align*}
Løsningen er $x = -11/2$ og $y = -25/2$.  Det vil si at vi kan
kombinere $\u$ og~$\w$ på denne måten for å komme oss til~$\b$:
\[
-\frac{11}{2} \u - \frac{25}{2} \w = \b
\]
Vi kom frem til~$\b$.  (Hurra!)  Kunne vi også kommet til et hvilket
som helst annet sted i~$\R^2$ ved å kombinere $\u$ og~$\w$?

Ja, det kunne vi.  Se på matrisene i gausselimineringen.  Om vi hadde
hatt andre tall på høyresiden fra begynnelsen av, så ville vi også ha
endt opp med andre tall på høyresiden til slutt, men tallene til
venstre for streken ville fremdeles vært de samme.  Og siden matrisen
på trappeform ikke inneholder noen rad med bare nuller til venstre for
streken, så må likningssystemet ha løsning uansett hva som står på
høyresiden.

Det betyr at vektorene $\u$ og~$\w$ til sammen lar oss bevege oss til
et hvilket som helst sted i~$\R^2$ -- i motsetning til $\u$ og~$\v$,
som ikke lot oss komme utenfor den ene rette linjen de begge ligger
på.

% TODO avslutte seksjonen

% TODO noe i R^3 også?  (evt oppg)


\section*{Lineærkombinasjoner}

Nå har vi snakket en del om «hvor det går an å komme ved hjelp av noen
vektorer» eller «måter å kombinere vektorer på» eller «å skalere
vektorene og så legge dem sammen».  Det er på tide at vi gir et
skikkelig navn til dette konseptet.
% Det vi får når vi skalerer opp
% noen vektorer med hvert sitt tall og så legger dem sammen kalles for
% en \emph{lineærkombinasjon} av vektorene.

\begin{defn}
La $\v_1$, $\v_2$, \ldots, $\v_n$ være vektorer i~$\R^m$, og la $c_1$,
$c_2$, \ldots, $c_n$ være skalarer.  Uttrykket
\[
c_1 \v_1 + c_2 \v_2 + \cdots c_n \v_n
\]
er en \defterm{lineærkombinasjon} av vektorene
$\v_1$, $\v_2$, \ldots, $\v_n$.
Skalarene
$c_1$, $c_2$, \ldots, $c_n$
kalles \defterm{vektene} i lineærkombinasjonen.
\end{defn}

% For eksempel er
% \[
% 7 \vvv{1}{-2}{4} - \frac{1}{2} \vvv{8}{0}{2}
% = \vvv{3}{-14}{27}
% \]
% en lineærkombinasjon av de to vektorene $(1,-2,4)$ og $(8,0,2)$
% i~$\R^3$, med vekter $7$ og~$-1/2$.

For eksempel er
\[
7 \u - 4 \v
\]
en lineærkombinasjon av vektorene $\u$ og~$\v$, med vekter $7$
og~$-4$.

\smallskip
Med begrepet «lineærkombinasjon» på plass i vokabularet kan vi enklere
beskrive det vi kom frem til i eksempelet i begynnelsen av kapitlet:
\begin{center}
\begin{tikzpicture}[scale=.6]
\draw[->] (-2.5,0) -- (4.5,0);
\draw[->] (0,-1.5) -- (0,3.5);
\foreach \x in {-2,...,-1,1,2,...,4}
\draw (\x,.1) -- (\x,-.1);% node[anchor=north] {$\x$};
\foreach \y in {-1,1,2,...,3}
\draw (.1,\y) -- (-.1,\y);% node[anchor=east] {$\y$};
\draw[->] (0,0) -- (4,2) node[anchor=west] {$\u$};
\draw[->] (0,0) -- (-2,-1) node[anchor=north] {$\v$};
\draw[->] (0,0) -- (-2,-.8) node[anchor=east] {$\w$};
\filldraw (3,-1) circle [radius=2pt] node[anchor=west] {$\b$};
\end{tikzpicture}
\end{center}
Nå kan vi si at vektoren~$\b$ kan skrives som en lineærkombinasjon av
$\u$ og~$\w$:
\[
\b = -\frac{11}{2} \u - \frac{25}{2} \w
\]
Men $\b$ kan ikke skrives som en lineærkombinasjon av $\u$ og~$\v$,
siden det ikke finnes noen tall $c$ og~$d$ som er slik at
\[
\b = c \u + d \v.
\]


% TODO mer tekst


\section*{Mengde utspent av vektorer}

Hvis vi har en samling med vektorer, så kan vi spørre \emph{Hvilke
  steder kan vi komme til ved å bruke disse vektorene?} -- eller, litt
mer presist: \emph{Hva er alle mulige lineærkombinasjoner av disse
  vektorene?}.  Svaret blir en mengde med vektorer, og den mengden
skal også få sitt eget navn.

\begin{defn}
La $\v_1$, $\v_2$, \ldots, $\v_n$ være vektorer i~$\R^m$.
\defterm{Mengden som er utspent av} disse vektorene består av
alle vektorer som kan skrives som lineærkombinasjoner av dem.
Vi bruker notasjonen
\[
\Sp \{ \v_1, \v_2, \ldots, \v_n \}
\]
for denne mengden.
\end{defn}

Hvis vi igjen ser på de tre vektorene $\u$, $\v$ og~$\w$ i $\R^2$ fra
begynnelsen av kapitlet, så ser vi at mengden utspent av $\u$ og~$\v$
består av alle vektorer som ligger på denne linjen:
\begin{center}
\begin{tikzpicture}[scale=.6]
\draw[->] (-4.5,0) -- (4.5,0);
\draw[->] (0,-2.5) -- (0,2.5);
\foreach \x in {-4,...,-1,1,2,...,4}
\draw (\x,.1) -- (\x,-.1);
\foreach \y in {-2,-1,1,2}
\draw (.1,\y) -- (-.1,\y);
\draw (-4.4,-2.2) -- (4.4,2.2);
\node[anchor=west] at (2.6,1) {$\Sp \{ \u, \v \}$};
\end{tikzpicture}
\end{center}
Mengden utspent av $\u$ og~$\w$, derimot, blir hele~$\R^2$:
\[
\Sp \{ \u, \w \} = \R^2
\]

% TODO mer?

% eksempler på Sp i R^2 og R^3 (og i R^4, ...?)


\kapittelslutt
