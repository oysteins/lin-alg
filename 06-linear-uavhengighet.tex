\input{kapittel}

\kapittel{6}{Lineær uavhengighet}
\label{ch:linear-uavhengighet}

Vi fortsetter med å utforske vektorer.  Nå har vi sett at vi kan
beskrive en uendelig stor mengde med vektorer ved å bare liste opp
noen få vektorer som \emph{spenner ut} hele mengden.  Men da kan vi
spørre om vi egentlig trenger å ha med alle vektorene i listen vår for
å beskrive akkurat den mengden, eller om noen av dem er overflødige.
Det spørsmålet er utgangspunktet for å snakke om lineær uavhengighet.
%TODO bedre intro?

\section*{Hvor mange vektorer trenger vi?}

I forrige kapittel så vi på disse to vektorene i~$\R^2$:
\begin{center}
\begin{tikzpicture}[scale=.6]
\draw[->] (-2.5,0) -- (4.5,0);
\draw[->] (0,-1.5) -- (0,3.5);
\foreach \x in {-2,...,-1,1,2,...,4}
\draw (\x,.1) -- (\x,-.1);% node[anchor=north] {$\x$};
\foreach \y in {-1,1,2,...,3}
\draw (.1,\y) -- (-.1,\y);% node[anchor=east] {$\y$};
\draw[->] (0,0) -- (4,2) node[anchor=west] {$\u = \vv{4}{2}$};
\draw[->] (0,0) -- (-2,-1) node[anchor=north] {$\v = \vv{-2}{-1}$};
\end{tikzpicture}
\end{center}
Mengden utspent av $\u$ og~$\v$ er denne rette linjen:
\begin{center}
\begin{tikzpicture}[scale=.6]
\draw[->] (-4.5,0) -- (4.5,0);
\draw[->] (0,-2.5) -- (0,2.5);
\foreach \x in {-4,...,-1,1,2,...,4}
\draw (\x,.1) -- (\x,-.1);
\foreach \y in {-2,-1,1,2}
\draw (.1,\y) -- (-.1,\y);
\draw (-4.4,-2.2) -- (4.4,2.2);
\node[anchor=west] at (2.6,1) {$\Sp \{ \u, \v \}$};
\end{tikzpicture}
\end{center}
Men hvis vi ser på mengden utspent av bare~$\u$, eller av bare~$\v$,
så får vi akkurat den samme linjen.  Mengden utspent av både $\u$
og~$\v$ er altså det samme som mengden utspent av bare én av dem:
\[
\Sp\{ \u, \v \} = \Sp\{ \u \} = \Sp\{ \v \}
\]
Hvis vi er interessert i mengden som utspennes av vektorene, så er det
altså unødvendig å ta med både $\u$ og~$\v$ -- det klarer seg med én
av dem.

Men med vektorene $\u$ og~$\w$ fra forrige kapittel forholder det seg
annerledes:
\begin{center}
\begin{tikzpicture}[scale=.6]
\draw[->] (-2.5,0) -- (4.5,0);
\draw[->] (0,-1.5) -- (0,3.5);
\foreach \x in {-2,...,-1,1,2,...,4}
\draw (\x,.1) -- (\x,-.1);% node[anchor=north] {$\x$};
\foreach \y in {-1,1,2,...,3}
\draw (.1,\y) -- (-.1,\y);% node[anchor=east] {$\y$};
\draw[->] (0,0) -- (4,2) node[anchor=west] {$\u = \vv{4}{2}$};
\draw[->] (0,0) -- (-2,-.8) node[anchor=north] {$\w = \vv{-2}{-4/5}$};
\end{tikzpicture}
\end{center}
Til sammen spenner disse to vektorene ut hele~$\R^2$:
\[
\Sp\{ \u, \w \} = \R^2
\]
Her er det essensielt at begge vektorene er med, for én av dem alene
spenner bare ut en linje:
\begin{center}
\begin{tikzpicture}[scale=.6]
\draw[->] (-4.5,0) -- (4.5,0);
\draw[->] (0,-2.5) -- (0,2.5);
\foreach \x in {-4,...,-1,1,2,...,4}
\draw (\x,.1) -- (\x,-.1);% node[anchor=north] {$\x$};
\foreach \y in {-2,-1,1,2}
\draw (.1,\y) -- (-.1,\y);% node[anchor=east] {$\y$};
\draw (-4,-2) -- (4,2) node[anchor=south] {$\Sp \{ \u \}$};
\draw (-4,-1.6) -- (4,1.6) node[anchor=west] {$\Sp \{ \w \}$};
\end{tikzpicture}
\end{center}

Så vi kan si at med de to vektorene $\u$ og~$\w$ er det slik at hver
av dem gir sitt eget, unike bidrag til den utspente mengden.  Med $\u$
og~$\v$, derimot, er det slik at hver av vektorene ikke bidrar med noe
mer enn det vi allerede får fra den andre.

\medskip%
Vi kan ta et litt mer komplisert eksempel også.  Da glemmer vi de
vektorene vi har sett på til nå, og så ser vi isteden på disse tre
vektorene i~$\R^3$:
\[
\a = \vvv{1}{1}{0}
\qquad
\b = \vvv{0}{1}{1}
\qquad
\c = \vvv{2}{5}{3}
\]
Prøv å se for deg disse tre vektorene som piler i et tredimensjonalt
koordinatsystem.  Hva er mengden som er utspent av dem?

Det er kanskje ikke så lett å se for seg.  Om vi først ser på hva som
blir utspent av bare én av vektorene, så blir det litt lettere.  Hver
av mengdene
\[
\Sp \{ \a \},\quad
\Sp \{ \b \},\quad\text{og}\quad
\Sp \{ \c \}
\]
er en rett linje.  Linjen utspent av~$\a$ er ganske lett å se for seg,
siden den holder seg i planet gitt ved første- og andreaksen:
\begin{center}
\begin{tikzpicture}[scale=.6]
\draw[->] (-2.5,0) -- (2.5,0);
\draw[->] (0,-1.8) -- (0,2.5);
\foreach \x in {-2,...,-1,1,2}
\draw (\x,.1) -- (\x,-.1);% node[anchor=north] {$\x$};
\foreach \y in {-1,1,2}
\draw (.1,\y) -- (-.1,\y);% node[anchor=east] {$\y$};
\draw (-2,-2) -- (2.2,2.2);
\node[anchor=west] at (1.5,1.5) {$\Sp\{ \a \}$};
\end{tikzpicture}
\end{center}
Linjen utspent av~$\b$ er tilsvarende grei: Den er i planet gitt ved
andre- og tredjeaksen.  Linjen utspent av~$\c$ er litt vanskeligere å
se for seg, for den står på skrå ut i rommet.

Nå kan du prøve å tenke på hva mengden
\[
\Sp \{ \a, \b \}
\]
blir for noe.  Se for deg de to linjene som er utspent av henholdsvis
$\a$ og~$\b$, og så kan du tenke at du legger et papirark oppå dem --
eller at du bruker de to linjene til å spenne ut en teltduk.  Da får
du et plan som går litt sånn på skrå gjennom rommet, og det planet er
mengden som er utspent av $\a$ og~$\b$.

Nå kan vi spørre: Ligger vektoren~$\c$ i planet som er utspent av $\a$
og~$\b$?  Du bør prøve å se omtrent hvordan det planet ser ut, og
hvordan~$\c$ ser ut, og vurdere om det ser ut som den er i nærheten av
planet eller ikke.  Det er ikke så lett, men du kan i hvert fall klare
å lage et omtrentlig bilde av situasjonen i hodet ditt.

Da finner du muligens ut at $\c$ ser ut til å være nær planet
$\Sp\{ \a, \b \}$, men det er vanskelig å si om den ligger nøyaktig i
det planet eller litt ved siden av.  For å finne ut av det, må vi se
på tallene.  Da finner vi ganske raskt ut at
\[
\c = 2\a + 3\b.
\]
Vi kan altså skrive $\c$ som en lineærkombinasjon av $\a$ og~$\b$, og
det betyr at $\c$ ligger i planet som er utspent av $\a$ og~$\b$:
\[
\c \in \Sp \{ \a, \b \}
\]
Dermed gir ikke $\c$ noe nytt bidrag til den utspente mengden, så
mengden utspent av $\a$, $\b$ og~$\c$ er bare planet som er utspent av
$\a$ og~$\b$:
\[
\Sp \{ \a, \b, \c \} = \Sp \{ \a, \b \}
\]
For å beskrive denne mengden, er det altså nok med de to vektorene
$\a$ og~$\b$ -- å ta med $\c$ i tillegg er overflødig.


\section*{Definisjon av lineær uavhengighet}

Hvis vi har en samling med vektorer som er sånn at hver eneste av dem
bidrar til å gjøre den utspente mengden større, så vil vi si at disse
vektorene er \emph{lineært uavhengige}.  Vi kan tenke på det som at
hver vektor gir sitt eget, uavhengige, bidrag til den utspente
mengden.

Nå kunne vi sagt at det er dette som er definisjonen av lineær
uavhengighet.  Men vi skal prøve å lage en litt bedre definisjon.  For
hva vil det egentlig si at hver vektor bidrar til å gjøre den utspente
mengden større?  Hvordan kan vi sjekke om det er sånn eller ikke?

Se for deg at vi har tre vektorer -- la oss kalle dem $\u$, $\v$
og~$\w$.  Hvordan kan vi sjekke at hver av dem gir et bidrag til den
utspente mengden?  Vi kan sjekke at
\begin{align*}
  \u &\text{ ikke er i $\Sp\{ \v, \w \}$,} \\
  \text{og at }
  \v &\text{ ikke er i $\Sp\{ \u, \w \}$,} \\
  \text{og at }
  \w &\text{ ikke er i $\Sp\{ \u, \v \}$.}
\end{align*}
Men det blir litt tungvint, og enda verre blir det hvis vi har mer enn
tre vektorer.

Men hva med det motsatte tilfellet, altså at de tre vektorene
\emph{ikke} er lineært uavhengige?  (Da vil vi forresten si at de er
lineært \emph{avhengige}.)  I så fall må det være en av dem som ikke
har noe uavhengig bidrag å komme med.  Det kan for eksempel være at
vektoren~$\w$ er med i mengden utspent av de to andre:
\[
\w \in \Sp\{ \u, \v \}
\]
Og hva betyr det?  Jo, det betyr at vi kan skrive $\w$ som en
lineærkombinasjon av $\u$ og~$\v$, altså at det finnes to tall $a$
og~$b$ slik at
\[
\w = a\u + b\v.
\]
Nå nærmer vi oss noe som kan egne seg i en definisjon!  Vi kan lage
likningen
\[
\w = x\u + y\v,
\]
og så kan vi si at hvis den har løsning, så er vektorene lineært
avhengige.

Men dette ser fremdeles ikke helt bra ut.  Her har vi satt $\w$ på den
ene siden av likningen, og $\u$ og~$\v$ på den andre -- uten at det
egentlig er noen grunn til det.  Det kunne jo like gjerne vært sånn at
det var $\u$ eller~$\v$ som var en lineærkombinasjon av de to andre.

Hva skal vi gjøre med det?  Vi kan flytte over $\w$ så vi får alle
vektorene på samme side.  Da ser likningen vår slik ut:
\[
x \u + y \v + (-1) \w = \0
\]
Så kan vi bytte ut $(-1)$ med en ny ukjent, $z$:
\[
x \u + y \v + z \w = \0
\]
Dersom vi isteden hadde startet med at det var $\u$ eller~$\v$ som var
en lineærkombinasjon av de andre, så kunne vi gått frem på samme måte
og kommet frem til akkurat den samme likningen.  Så i alle tilfeller
der vektorene er lineært avhengige, så kan vi si at \emph{denne}
likningen må ha løsning.

Men vent nå litt.  Uansett hva vektorene $\u$, $\v$ og~$\w$ skulle
finne på å være, så har likningen
\[
x \u + y \v + z \w = \0
\]
minst én løsning: Vi kan jo bare sette alle de ukjente -- altså $x$,
$y$ og~$z$ -- til~$0$.

Hm.  Men det er ikke den samme løsningen som vi hadde da vi antok at
$\w$ var en lineærkombinasjon av $\u$ og~$\v$, for da hadde vi $z=-1$.

Og da kommer vi til selve poenget.  Løsningen der vi setter alle de
ukjente til~$0$ kaller vi gjerne for den \emph{trivielle} løsningen.
Men hvis likningen
\[
x \u + y \v + z \w = \0
\]
også har en \emph{annen} løsning enn den trivielle, da vet vi at én av
vektorene kan skrives som en lineærkombinasjon av de to andre, og
dermed er de tre vektorene lineært avhengige.

For å definere lineær \emph{u}avhengighet, vil vi rett og slett snu om
på det: Dersom likningen \emph{kun} har den trivielle løsningen, så er
vektorene lineært uavhengige.

Nå har vi snakket om tre vektorer, men generelt kan vi snakke om
lineær (u)avhengighet for en liste med hvor mange vektorer som helst.
Vi skriver opp en generell definisjon:

\begin{defn}
La $\V{v}_1$, $\V{v}_2$, \ldots, $\V{v}_n$ være vektorer i $\R^m$.
Disse vektorene er \defterm{lineært uavhengige} dersom likningen
\[
\V{v}_1 \cdot x_1 + \V{v}_2 \cdot x_2 + \cdots + \V{v}_n \cdot x_n = \V{0}
\]
ikke har andre løsninger enn den trivielle løsningen
$x_1 = x_2 = \cdots = x_n = 0$.

I motsatt tilfelle kalles de \defterm{lineært avhengige}.
\end{defn}

% TODO enkelt eksempel i \R^2 eller \R^3?

\begin{ex}
Her er tre vektorer i~$\R^4$:
\[
\V{u} = \vvvv{3}{9}{3}{3},\qquad
\V{v} = \vvvv{2}{7}{2}{4}\qquad\text{og}\qquad
\V{w} = \vvvv{8}{31}{12}{22}
\]
Er de lineært uavhengige?

For å svare på det, bruker vi definisjonen av lineær uavhengighet.  Vi
må finne ut om likningen
\[
\V{u} \cdot x + \V{v} \cdot y + \V{w} \cdot z = \V{0},
\]
har noen ikketriviell løsning.
Denne likningen er det samme som følgende likningssystem:
\[
\systeme{
  3x + 2y + 8z = 0,
  9x + 7y + 31z = 0,
  3x + 2y + 12z = 0,
  3x + 4y + 22z = 0
}
\]
Det løser vi på vanlig måte med gausseliminasjon:
\[
\begin{amatrix}{3}
3 & 2 & 8  & 0 \\
9 & 7 & 31 & 0  \\
3 & 2 & 12 & 0  \\
3 & 4 & 22 & 0 
\end{amatrix}
\sim
\begin{amatrix}{3}
3 & 2 & 8 & 0  \\
0 & 1 & 7 & 0  \\
0 & 0 & 4 & 0  \\
0 & 0 & 0 & 0 
\end{amatrix}
\]
Vi kunne fortsatt videre til redusert trappeform, men allerede her er
det tydelig at vi får kun én løsning: $x = y = z = 0$.  Dette betyr at
vektorene $\V{u}$, $\V{v}$ og~$\V{w}$ er lineært uavhengige.
\end{ex}

%TODO noe mer?  avslutning av kapitlet?


\kapittelslutt
