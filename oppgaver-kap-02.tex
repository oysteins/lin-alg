% -*- mode: LaTeX; TeX-master: "lin-alg"; -*-
\oppgaver{2}

% TODO flere enkle oppgaver?

\begin{oppgave}
Hvilke av disse matrisene er på trappeform?  Hvilke av dem er på
redusert trappeform?
\begin{punkt}
$
\begin{bmatrix}
1 & 5 & 0 & 0 \\
0 & 0 & 0 & 1
\end{bmatrix}
$
\end{punkt}
\begin{punkt}
$
\begin{bmatrix}
1 & 0 \\
0 & 1 \\
0 & -1
\end{bmatrix}
$
\end{punkt}
\begin{punkt}
$
\begin{bmatrix}
0 & 2 & 1 \\
0 & 0 & 4 \\
0 & 0 & 0
\end{bmatrix}
$
\end{punkt}
\begin{punkt}
$
\begin{bmatrix}
0 & 0 & 0 \\
0 & 0 & 0 \\
0 & 0 & 0
\end{bmatrix}
$
\end{punkt}
\end{oppgave}
\begin{losning}
Matrise (a), (c) og (d) er på trappeform;
(a) og~(d) er på redusert trappeform.
\end{losning}

\begin{oppgave}
Løs likningssystemene:
\begin{punkt}
$
\systeme{
  2x - 4y + 9z = -38,
  4x - 3y + 8z = -26,
 -2x + 4y - 2z =  17
}
$
\end{punkt}
\begin{punkt}
% 1 0 0 4
% 0 1 2 0
% 0 0 0 0
% ---
% 1 3 6 4
% 0 2 4 0
% 0 0 0 0
% ---
% 1 3  6 4
% 2 8 16 8
% 0 0  0 0
% ---
% 1 3  6 4
% 2 8 16 8
% 2 6 12 8
$
\systeme{
  x + 3y +  6z = 4,
 2x + 8y + 16z = 8,
 2x + 6y + 12z = 8
}
$
\end{punkt}
\begin{punkt}
$
\systeme{
	x  + 2y - z = 1,
	2x + 3y - z = -1,
	3x + 4y - z = 1
}
$
\end{punkt}
\end{oppgave}

\begin{losning}
Her er det selvfølgelig lurt å bruke gausseliminasjon, men du har litt
valgfrihet når det gjelder hvordan du gjør det.  Du velger selv hvilke
radoperasjoner du gjør, og i hvilken rekkefølge, så lenge du klarer å
ende opp med en matrise på trappeform til slutt.  Du velger også selv
om du vil fortsette helt til redusert trappeform, eller om du nøyer
deg med bare trappeform.
\begin{punkt}
Her er én måte å gausseliminere frem til redusert trappeform:
\begin{align*}
\begin{amatrix}{3}
2 & -4 & 9 & -38 \\
4 & -3 & 8 & -26 \\
-2 & 4 & -2 & 17
\end{amatrix}
&
\roweq
\begin{amatrix}{3}
2 & -4 & 9 & -38 \\
0 & 5 & -10 & 50 \\
0 & 0 & 7 & -21
\end{amatrix}
\\
&
\roweq
\begin{amatrix}{3}
2 & -4 & 9 & -38 \\
0 & 1 & -2 & 10 \\
0 & 0 & 1 & -3
\end{amatrix}
\\
&
\roweq
\begin{amatrix}{3}
2 & -4 & 0 & -11 \\
0 & 1 & 0 & 4 \\
0 & 0 & 1 & -3
\end{amatrix}
\\
&
\roweq
\begin{amatrix}{3}
2 & 0 & 0 & 5 \\
0 & 1 & 0 & 4 \\
0 & 0 & 1 & -3
\end{amatrix}
\\
&
\roweq
\begin{amatrix}{3}
1 & 0 & 0 & \frac{5}{2} \\
0 & 1 & 0 & 4 \\
0 & 0 & 1 & -3
\end{amatrix}
\end{align*}
Vi ser at løsningen blir:
$x=\frac{5}{2}$, $y=4$ og $z=-3$.
\end{punkt}

\begin{punkt}
Gausseliminering til redusert trappeform:
\begin{align*}
\begin{amatrix}{3}
1 & 3 & 6 & 4 \\
2 & 8 & 16 & 8 \\
2 & 6 & 12 & 8
\end{amatrix}
&
\roweq
\begin{amatrix}{3}
1 & 3 & 6 & 4 \\
0 & 2 & 4 & 0 \\
0 & 0 & 0 & 0
\end{amatrix}
\\
&
\roweq
\begin{amatrix}{3}
1 & 3 & 6 & 4 \\
0 & 1 & 2 & 0 \\
0 & 0 & 0 & 0
\end{amatrix}
\\
&
\roweq
\begin{amatrix}{3}
1 & 0 & 0 & 4 \\
0 & 1 & 2 & 0 \\
0 & 0 & 0 & 0
\end{amatrix}
\end{align*}
Her må vi tenke oss litt om når vi skal skrive opp løsningen.  La oss
se hva den siste matrisen egentlig gir oss når vi skriver den om til
et likningssystem igjen:
\[
\left\{
\begin{aligned}
x &= 4 \\
y + 2z &= 0 \\
0 &= 0
\end{aligned}
\right.
\]
Den siste likningen sier bare at $0=0$, og det er sant uansett hva
$x$, $y$ og~$z$ måtte være, så den kan vi sløyfe.  Hvis vi skriver om
likningen med $y$ og~$z$ slik at $y$ står alene på venstre side, så
får vi:
\[
\left\{
\begin{aligned}
x &= 4 \\
y &= -2z
\end{aligned}
\right.
\]
Så her kommer vi frem til en løsning for $x$ og~$y$, men hva skal $z$
være, da?  Vel, vi kan faktisk la~$z$ være hva som helst.  Uansett
hvilken verdi vi setter inn for~$z$, så kan vi sette $y$ til å være
$-2z$, og så får vi en løsning av likningssystemet.

Hvis vi for eksempel velger $z=3$, så får vi denne løsningen:
\[
\left\{
\begin{aligned}
x &= 4 \\
y &= -6 \\
z &= 3
\end{aligned}
\right.
\]
Hvis vi velger $z=0$, så får vi denne:
\[
\left\{
\begin{aligned}
x &= 4 \\
y &= 0 \\
z &= 0
\end{aligned}
\right.
\]
Så totalt sett kan vi si at alle løsningene er:
\[
\left\{
\begin{aligned}
x &= 4 \\
y &= -2z \\
z &\text{ er et hvilket som helst tall}
\end{aligned}
\right.
\]
Dette likningssystemet har altså uendelig mange løsninger.
Vi skal se mer til slike likningssystemer i neste kapittel.
\end{punkt}

\begin{punkt}
Gausseliminering til redusert trappeform:
\begin{align*}
\begin{amatrix}{3}
1 & 2 & -1 & 1 \\
2 & 3 & -1 & -1 \\
3 & 4 & -1 & 1
\end{amatrix}
&
\roweq
\begin{amatrix}{3}
1 & 2 & -1 & 1 \\
0 & -1 & 1 & -3 \\
0 & -2 & 2 & -2
\end{amatrix}
\\
&
\roweq
\begin{amatrix}{3}
1 & 2 & -1 & 1 \\
0 & 1 & -1 & 3 \\
0 & -2 & 2 & -2
\end{amatrix}
% \\
% &
% \roweq
% \begin{amatrix}{3}
% 1 & 2 & -1 & 1 \\
% 0 & 1 & -1 & 3 \\
% 0 & 0 & 0 & 4
% \end{amatrix}
\\
&
\roweq
\begin{amatrix}{3}
1 & 0 & 1 & -5 \\
0 & 1 & -1 & 3 \\
0 & 0 & 0 & 4
\end{amatrix}
\end{align*}
Her, som i del~(b), er også matrisen vi ender opp med litt pussig.  La
oss skrive opp det tilsvarende likningssystemet:
\[
\left\{
\begin{aligned}
x + z &= -5 \\
y - z &= 3 \\
0 &= 4
\end{aligned}
\right.
\]
Den siste likningen sier at $0=4$.  Men kan det stemme, da?  Nei, det
kan det ikke.  Uansett hva vi måtte sette $x$, $y$ og~$z$ til å være,
så kan det aldri bli sant at $0=4$.

Men likningssystemet krever altså at vi \emph{skal} finne $x$, $y$
og~$z$ som er slik at $0=4$.  Eller med andre ord: En løsning av
systemet må bestå av tall vi kan sette inn for $x$, $y$ og~$z$ slik at
$0$ blir~$4$.  Men sånne tall finnes ikke, og det betyr at
likningssystemet ikke har noen løsning.
Vi skal se mer til slike likningssystemer i neste kapittel.
\end{punkt}
\end{losning}


% TODO er matrisene radekvivalente?
% flere oppgaver
% matriser i forskjellige størrelser (noen med ikke-kvadratisk koeff-matrise?)
% én der en av matrisene er på (redusert) trappeform
% én der man kommer fra den ene til den andre med et par radoperasjoner
% én der man ganske lett kan se at én variabel får ulik verdi i de to systemene

\begin{oppgave}
Er disse matrisene radekvivalente?
\[
\begin{amatrix}{2}
2 & -2 & 7 \\
4 & -3 & 11 \\
-2 & -1 & 2
\end{amatrix}
\qquad\text{og}\qquad
\begin{amatrix}{2}
0 & -3 & 9 \\
8 & -6 & 22 \\
6 & -7 & 24
\end{amatrix}
\]
\end{oppgave}

\begin{losning}
Ja, matrisene er radekvivalente, for vi kan komme fra den ene til den
andre ved å gjøre tre radoperasjoner (legge til tredje rad i første
rad, legge til to ganger andre rad i tredje rad, og gange andre rad
med to):
\begin{align*}
\begin{amatrix}{2}
2 & -2 & 7 \\
4 & -3 & 11 \\
-2 & -1 & 2
\end{amatrix}
&
\roweq
\begin{amatrix}{2}
0 & -3 & 9 \\
4 & -3 & 11 \\
-2 & -1 & 2
\end{amatrix}
\\
&
\roweq
\begin{amatrix}{2}
0 & -3 & 9 \\
4 & -3 & 11 \\
6 & -7 & 24
\end{amatrix}
\\
&
\roweq
\begin{amatrix}{2}
0 & -3 & 9 \\
8 & -6 & 22 \\
6 & -7 & 24
\end{amatrix}
\end{align*}
\end{losning}


\begin{oppgave}
Er disse matrisene radekvivalente?
\begin{gather*}
\begin{amatrix}{4}
5 & 6 & 8 & -11 & -5 \\
1 & -3 & 7 & -1 & 9 \\
-2 & -3 & 2 & 5 & 6 \\
3 & 0 & 6 & -6 & 3
\end{amatrix}
\hspace{80pt}
\\
\hspace{55pt}\text{og}\qquad
\begin{amatrix}{4}
1 & 0 & 0 & 0 & 13 \\
0 & -3 & 0 & 1 & 8 \\
0 & 0 & 1 & 0 & 0 \\
0 & 0 & 0 & 1 & 6
\end{amatrix}
\end{gather*}
\end{oppgave}

\begin{losning}
Her kan det bli mye jobb å komme fra den ene matrisen til den andre
ved hjelp av radoperasjoner.

Men vi ser at den andre matrisen er på trappeform (og nesten på
redusert trappeform), så det er lett å lese ut en løsning fra den.
Hvis vi kaller de fire ukjente for $x$, $y$, $z$ og~$æ$, så kan vi
skrive om matrisen til dette likningssystemet:
\[
\left\{
\begin{aligned}
x &= 13 \\
-3y + æ &= 8 \\
z &= 0 \\
æ &= 6
\end{aligned}
\right.
\]
Dermed får vi følgende løsning:
\[
\left\{
\begin{aligned}
x &= 13 \\
y &= -\frac{2}{3} \\
z &= 0 \\
æ &= 6
\end{aligned}
\right.
\]

%TODO
\end{losning}
% TODO heller lage oppgave der de ikke er radekvivalente?  eller i tillegg?


\begin{oppgave}
Løs likningssystemet:
\[
\left\{
\begin{aligned}
    xy -  3xz +  yz &= -5 \\
  -2xy +  7xz - 2yz &= 25 \\
   6xy - 18xz + 7yz &= 0
\end{aligned}
\right.
% \systeme{
%     xy -  3xz +  yz = -5,
%   -2xy +  7xz - 2yz = 25,
%    6xy - 18xz + 7yz = 0
% }
\]
\end{oppgave}

\begin{losning}
% xy = a
% xz = b
% yz = c
%%%%
% z = c/y
% x(c/y) = b
% y = x * c/b
% x (x * c/b) = a
% x^2 = ab/c
% y^2 = x^2 * c^2/b^2 = ab/c * c^2/b^2 = ac/b
% z^2 = c^2/y^2 = c^2/(ac/b) = bc^2/ac = bc/a
%%%%
% a 1 0 1  10
% b 0 1 1  15
% c 1 1 1  30
%%%%
% x^2 = 10*15/30 = 5
% x = \sqrt{5}
% y = \sqrt{5} * c/b = 2 \sqrt{5}
% z = c / y = 30 / (2\sqrt{5}) = 3 \sqrt{5}
Det første vi legger merke til her, er at dette \emph{ikke} er et
lineært likningssystem.  Så selv om vi har en flott metode --
gausseliminasjon -- for å løse alle lineære likningssystemer, så
hjelper ikke den metoden oss her.

Eller gjør den det?  Selv om systemet ikke er lineært, så ligner det
jo ganske mye på et lineært system.  Hvis hvert av uttrykkene $xy$,
$xz$ og~$yz$ bare hadde vært én ukjent istedenfor et produkt av to
ukjente, så ville systemet vært lineært.

Ikke verre?  Det kan vi faktisk fikse ved å trikse litt.  Vi lager tre
nye variabler:
\begin{align*}
  u &= xy \\
  v &= xz \\
  w &= yz
\end{align*}
Dermed kan systemet skrives slik:
\[
\systeme{
    u -  3v +  w = -5,
  -2u +  7v - 2w = 25,
   6u - 18v + 7w = 0
}
\]
\emph{Et voilà} -- det er lineært!

Så er det bare å gå i gang med gausseliminasjon, og da finner du at
løsningen er:
\[
\left\{
\begin{aligned}
  u &= 10 \\
  v &= 15 \\
  w &= 30
\end{aligned}
\right.
\]
Men det var jo ikke egentlig $u$, $v$ og~$w$ vi skulle finne -- det
var $x$, $y$ og~$z$!  Vi må huske på at vi innførte de nye variablene
$u$, $v$ og~$w$ til å stå for henholdsvis $xy$, $xz$ og~$yz$, og det
betyr at vi nå vet at:
\[
\left\{
\begin{aligned}
 xy &= 10 \\
 xz &= 15 \\
 yz &= 30
\end{aligned}
\right.
\]
Nå må vi prøve å rydde opp i dette sånn at vi får $x$, $y$ og~$z$ hver
for seg.  Det kan vi for eksempel gjøre sånn som dette: Fra den siste
likningen får vi at $z = 30/y$.  Det kan vi sette inn i den midterste
likningen.  Da får vi
\[
x \cdot \frac{30}{y} = 15,
\]
som vi kan forenkle til $y = 2x$.  Så kan vi sette inn det i den
første likningen og få
\[
x \cdot (2x) = 10,
\]
som vil si at $x^2 = 5$.
La oss oppsummere det vi har funnet ut:
\[
\left\{
\begin{aligned}
x^2 &= 5 \\
y &= 2x \\
z &= \frac{15}{x}
\end{aligned}
\right.
\]
Nå blir det to muligheter for~$x$, nemlig $\sqrt{5}$ og $-\sqrt{5}$.
Dermed får vi to løsninger:
\[
\left\{
\begin{aligned}
x &= \sqrt{5} \\
y &= 2 \sqrt{5} \\
z &= 3 \sqrt{5}
\end{aligned}
\right.
\qquad\qquad
\left\{
\begin{aligned}
x &= -\sqrt{5} \\
y &= -2 \sqrt{5} \\
z &= -3 \sqrt{5}
\end{aligned}
\right.
\]
\end{losning}


\begin{oppgave}
% TODO fiks
La $(1,2)$, $(2,3)$ og $(3,5)$ være tre punkter i planet. Vi skal
finne et andregradspolynom $ax^2 + bx + c$ slik at grafen går gjennom
de tre punktene.
\begin{center}
\begin{tikzpicture}[scale=.75]
\draw[->] (-1.5,0) -- (5.6,0);
\draw[->] (0,-0.5) -- (0,5.5);
\foreach \x in {-1,1,2,3,4,5}
\draw (\x cm,1pt) -- (\x cm,-1pt) node[anchor=north] {$\x$};
\foreach \y in {1,2,3,4,5}
\draw (1pt,\y cm) -- (-1pt,\y cm) node[anchor=east] {$\y$};
\filldraw (1,2) circle [radius=2pt] node[anchor=west] {$(1,2)$};
\filldraw (2,3) circle [radius=2pt] node[anchor=west] {$(2,3)$};
\filldraw (3,5) circle [radius=2pt] node[anchor=west] {$(3,5)$};
\end{tikzpicture}
\vspace{-5pt}
\end{center}
\begin{punkt}
Sett opp et lineært likningssystem for $a$, $b$ og~$c$.
\end{punkt}
\begin{punkt}
Løs systemet, og finn andregradspolynomet som går gjennom alle punktene.
\end{punkt}
\begin{punkt}
Sjekk at svaret ditt i \textbf{b)} er riktig.
\end{punkt}
\end{oppgave}

\begin{losning}
	\begin{punkt}
		Kravene $p(1)=2$, $p(2)=3$ og $p(3)=5$ gir følgende likningssystem:
		$$
		\systeme{
			a+b+c = 2,
			4a+2b+c = 3,
			9a+3b+c = 5
		}
		$$
		
	\end{punkt}
	\begin{punkt}
		Løsningen er $a=\frac{1}{2}$, $b=-\frac{1}{2}$ og~$c=2$.
	\end{punkt}
	
	\begin{punkt}
		Sett inn $1$, $2$ og~$3$ i polynomet $p(x)=\frac{1}{2}x^2-\frac{1}{2}x+2$ for å se at kravene i \textbf{a)} er oppfylt.  
	\end{punkt}
\end{losning}


\begin{oppgave}
% TODO fiks (eller sløyf)
Vis at følgende påstander er sanne for alle matriser $M$, $N$ og~$L$:
\begin{punkt}
$M \roweq M$.
\end{punkt}
\begin{punkt}
Hvis $M \roweq N$, så: $N \roweq M$.
\end{punkt}
\begin{punkt}
Hvis $M \roweq L$ og $L \roweq N$, så: $M \roweq N$.
\end{punkt}
\end{oppgave}

\begin{losning}
\begin{punkt}
	$M$ er trivielt radekvivalent med $M$.
\end{punkt}
\begin{punkt}
	Hint: Alle radoperasjoner er reversible; multiplisere en rad med et ikke-null tall $c$ kan reverseres ved å multiplisere samme rad med $\frac{1}{c}$; bytte om på to rader kan reverseres ved å bytte om på radene igjen; å legge til et multiplum av en rad til en annen kan reverseres ved å trekke fra det som ble lagt til. Dersom $M\roweq N$ betyr det at vi har gjort et endelig antall radoperasjoner $\text{O}_1,\dots,\text{O}_n$ for å lage $N$ fra $M$. Klarer du, ut ifra dette, å finne et endelig antall radoperasjoner som lager $M$ fra $N$?
\end{punkt}
\begin{punkt}
	Hint: Vi antar at det finnes et endelig antall radoperasjoner $\text{O}_1,\dots,\text{O}_n$ som lager $L$ fra $M$, og at det finnes et endelig antall radoperasjoner $\text{O}_{n+1},\dots,\text{O}_{n+m}$ som lager $N$ fra $L$. Klarer du, ut ifra dette, å finne et endelig antall radoperasjoner som lager $N$ fra $M$?
\end{punkt}
\end{losning}
