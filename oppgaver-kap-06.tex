% -*- mode: LaTeX; TeX-master: "lin-alg"; -*-
\oppgaver{6}

\begin{oppgave}
Se på disse vektorene i~$\R^2$:
\begin{center}
\begin{tikzpicture}
\draw[->] (-2.5,0) -- (4,0);
\draw[->] (0,-1.5) -- (0,2.5);
\draw[->] (0,0) -- (3,1) node[anchor=west] {$\u$};
\draw[->] (0,0) -- (-2,1) node[anchor=north] {$\v$};
\draw[->] (0,0) -- (1,-1) node[anchor=north] {$\w$};
\end{tikzpicture}
\end{center}
Er $\u$ og~$\v$ lineært uavhengige?
Er $\u$, $\v$ og~$\w$ lineært uavhengige?
\end{oppgave}

\begin{losning}
Vektorene $\u$ og~$\v$ er lineært uavhengige,
men $\u$, $\v$ og~$\w$ er ikke.
\end{losning}


\begin{oppgave}
Er disse to vektorene lineært uavhengige?
\[
\V{v}_1 = \vvv{8}{2}{-12}
\qquad\text{og}\qquad
\V{v}_2 = \vvv{4}{1}{-6}
\]
\end{oppgave}

\begin{losning}
Siden vi har $\V{v}_1 = 2 \cdot \V{v}_2$, får vi at nullvektoren kan
skrives som en ikketriviell lineærkombinasjon av $\V{v}_1$ og~$\V{2}$
på denne måten:
\[
1 \cdot \V{v}_1 - 2 \cdot \V{v}_2 = \V{0}
\]
Vektorene $\V{v}_1$ og $\V{v}_2$ er derfor lineært avhengige.
\end{losning}


\begin{oppgave}
Er disse vektorene lineært uavhengige?
\[
\V{v}_1 = \vvvv{5}{10}{5}{0},\qquad
\V{v}_2 = \vvvv{3}{7}{5}{4}\qquad\text{og}\qquad
\V{v}_3 = \vvvv{2}{6}{6}{8}
\]
\end{oppgave}

\begin{losning}
Vi må løse likningen
\[
\v_1 x_1 + \v_2 x_2 + \v_3 x_3 = \0.
\]
Gausseliminasjon gir:
\[
\begin{amatrix}{3}
5 & 3 & 2 & 0 \\
10 & 7 & 6 & 0 \\
5 & 5 & 6 & 0 \\
0 & 4 & 8 & 0
\end{amatrix}
\sim
\begin{amatrix}{3}
5 & 3 & 2 & 0 \\
0 & 1 & 2 & 0 \\
0 & 0 & 0 & 0 \\
0 & 0 & 0 & 0
\end{amatrix}
\]
Det blir ikke noe pivotelement i tredje kolonne.  Vi får altså en fri
variabel, så vi har uendelig mange løsninger.  Dermed er $\V{v}_1$,
$\V{v}_2$ og $\V{v}_3$ lineært avhengige.
\end{losning}


\begin{oppgave}
Finn ut om hver av disse påstandene er sann eller ikke:
\begin{punkt}
To vektorer $\u$ og $\v$ er lineært avhengige hvis og bare hvis $\u$
er i mengden utspent av~$\v$.
\end{punkt}
\begin{punkt}
To vektorer $\u$ og $\v$ er lineært avhengige hvis og bare hvis en av
dem er lik en skalar ganger den andre.
\end{punkt}
\begin{punkt}
Tre vektorer $\u$, $\v$ og~$\w$ er lineært avhengige hvis og bare
hvis $\u$ kan skrives som en lineærkombinasjon av $\v$ og~$\w$.
\end{punkt}
\end{oppgave}

\begin{losning}
\begin{punkt}
Usant.%TODO
\end{punkt}

\begin{punkt}
Anta først at $\V{u}$ og~$\V{v}$ er lineært avhengige.  Da finnes
to tall $a$ og~$b$ slik at
\[
\V{u} \cdot a + \V{v} \cdot b = \V{0}
\]
og minst én av $a$ og~$b$ er ulik~$0$.  Hvis $a \ne 0$, får vi
\[
\V{u} = - \frac{b}{a} \V{v}.
\]
Hvis $b \ne 0$, får vi
\[
\V{v} = - \frac{a}{b} \V{u}.
\]
I begge tilfeller har vi at én av vektorene er en skalar ganger den
andre.

Nå viser vi den motsatte implikasjonen.  Anta derfor at én av vektorene
er en skalar ganger den andre.  Hvis $\V{u} = c \cdot \V{v}$ for en
skalar~$c$, så får vi:
\[
\V{u} \cdot 1 + \V{v} \cdot (-c) = \V{0}.
\]
Hvis $\V{v} = d \cdot \V{u}$ for en skalar~$d$, så får vi:
\[
\V{u} \cdot d + \V{v} \cdot (-1) = \V{0}.
\]
I begge tilfeller har vi en ikketriviell løsning av likningen
$\V{u} \cdot x_1 + \V{v} \cdot x_2 = \V{0}$,
og det betyr at vektorene er lineært avhengige.

Det var slutten på beviset.  Nå kan vi tenke litt på hva dette
egentlig betyr.

Det vi har sett nå, er at to vektorer er lineært uavhengige hvis og
bare hvis de ikke ligger på en rett linje gjennom origo.

Det kan være spesielt interessant å se på to vektorer i $\R^2$.  Da er
det to muligheter: Enten er de lineært avhengige, eller så utspenner
de hele~$\R^2$.
\end{punkt}

\begin{punkt}
Denne påstanden er usann,%TODO
\end{punkt}
\end{losning}


\begin{oppgave}
Anta at du har tre vektorer $\u$, $\v$ og~$\w$, som er slik at
\begin{align*}
  &\text{$\u$ og~$\v$ er lineært uavhengige,} \\
  \text{og }
  &\text{$\u$ og~$\w$ er lineært uavhengige,} \\
  \text{og }
  &\text{$\v$ og~$\w$ er lineært uavhengige.}
\end{align*}
Kan du da konkludere med at $\u$, $\v$ og~$\w$ er lineært uavhengige?
\end{oppgave}

\begin{losning}
Nei.  Her er et enkelt eksempel i~$\R^2$:
\[
\u = \vv{1}{0}
\qquad
\v = \vv{0}{1}
\qquad
\w = \vv{1}{1}
\]
\end{losning}


\begin{oppgave}
Er disse vektorene lineært uavhengige?
\[
\V{v}_1 = \vvv{8}{7}{4},\quad
\V{v}_2 = \vvv{14}{-2}{5},\quad
\V{v}_3 = \vvv{3}{1}{0},\quad
\V{v}_4 = \vvv{7}{5}{11}\quad
\]
\end{oppgave}

\begin{losning}
For å sjekke om vektorene er lineært uavhengige, kan vi gausseliminere
denne matrisen:
\[
\begin{amatrix}{4}
8 & 14 & 3 & 7 & 0 \\
7 & -2 & 1 & 5 & 0 \\
4 & 5 & 0 & 11 & 0
\end{amatrix}
\]
Men vi trenger ikke egentlig å utføre gausseliminasjonen.  Vi ser med
en gang at uansett hva som skjer, så kan vi ikke få mer enn tre
pivotelementer (ett i hver rad).  Dermed kan det ikke bli
pivotelementer i alle de fire kolonnene til venstre for streken, så vi
må få minst én fri variabel, og det betyr at vektorene $\V{v}_1$,
$\V{v}_2$, $\V{v}_3$ og $\V{v}_4$ er lineært avhengige.
\end{losning}


\begin{oppgave}
Du har $n$ vektorer $\V{v}_1$, $\V{v}_2$, \ldots, $\V{v}_n$ i $\R^m$.
Vis at dersom
\begin{enumerate}
\item en av vektorene er en lineærkombinasjon av de andre, eller
\item en av vektorene er $\V0$, eller
\item $n > m$,
\end{enumerate}
så er vektorene lineært avhengige.
\end{oppgave}

\begin{losning}
Anta først at én vektor~$\V{v}_k$ er en lineærkombinasjon av de andre:
\[
\V{v}_k = \sum_{i \ne k} a_i \V{v}_i
\]
Da kan vi sette $a_k = -1$ og få:
\[
\sum_{i = 1}^n a_i \V{v}_i = \V{0}
\]
Her har vi skrevet nullvektoren som en ikketriviell lineærkombinasjon
av vektorene våre (vi vet ikke hva alle $a_i$-ene er, men vi vet i
hvert fall at én av dem, $a_k$, ikke er~$0$).  Det betyr at vektorene
er lineært avhengige.

Nå går vi videre til å se på den andre antagelsen i oppgaven, så vi
antar at én av vektorene i listen er nullvektoren.  Hvis
$\V{v}_k = \V{0}$, så kan vi definere $n$ tall $a_1$, $a_2$, \ldots,
$a_n$ ved:
\[
a_i =
\begin{cases}
0 & \text{hvis $i \ne k$} \\
1 & \text{hvis $i = k$}
\end{cases}
\]
Da får vi at
\[
\sum_{i = 1}^n a_i \V{v}_i = \V{0},
\]
og vektorene er lineært avhengige.

Til slutt ser vi på den tredje antagelsen.  Akkurat som i forrige
oppgave får vi her at når vi gausseliminerer matrisen
\[
\begin{amatrix}{4} \V{v}_1 & \V{v}_2 & \ldots & \V{v}_n & \0 \end{amatrix},
% TODO kan vi bruke denne matrisenotasjonen her?
\]
så får vi maksimalt $m$ pivotelementer (ett i hver rad), men vi
trenger $n$ pivotelementer (ett i hver kolonne til venstre for
streken) for at vektorene skal være lineært uavhengige.  Når $n > m$
går ikke det an, så da er vektorene lineært avhengige.
\end{losning}


\begin{oppgave}
Du har $n$ vektorer $\V{v}_1$, $\V{v}_2$, \ldots, $\V{v}_n$ i $\R^m$.
Vis at disse vektorene er lineært uavhengige hvis og bare hvis ingen
av dem kan skrives som en lineærkombinasjon av de andre.
\end{oppgave}

\begin{losning}
Påstanden i oppgaven er det samme som å si at vektorene er lineært
avhengige hvis og bare hvis en av dem kan skrives som en
lineærkombinasjon av de andre.

I forrige oppgave viste vi at dersom en av vektorene er en
lineærkombinasjon av de andre, så er de lineært avhengige.  Det
gjenstår å vise at dersom vektorene er lineært avhengige, så er en av
dem en lineærkombinasjon av de andre.

Anta at vektorene er lineært avhengige, altså at vi har
\[
a_1 \V{v}_1 + a_2 \V{v}_2 + \cdots + a_n \V{v}_n = \V{0},
\]
der minst én av $a_i$-ene er ulik~$0$.  Velg en~$k$ slik at $a_k \ne 0$.
Da har vi:
\[
a_k \V{v}_k = \sum_{i \ne k} (-a_i) \V{v}_i
\]
Siden $a_k \ne 0$ kan vi dele på $a_k$ og få:
\[
\V{v}_k = \sum_{i \ne k} \frac{-a_i}{a_k} \cdot \V{v}_i
\]
Dermed er vektoren $\V{v}_k$ en lineærkombinasjon av de andre
vektorene i listen.
\end{losning}

%%%% TODO resultat om n vektorer i R^n
