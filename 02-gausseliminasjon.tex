\input{kapittel}

\kapittel{2}{Gausseliminasjon}
\label{ch:gausseliminasjon}

Nå skal vi formalisere ideene fra
kapittel~\ref{ch:lineare-likningssystemer}.  Vi skal se hvordan vi
kan løse et hvilket som helst lineært likningssystem ved å skrive om
totalmatrisen til systemet etter bestemte regler.

Reglene for hvordan totalmatrisen kan skrives om kalles
\emph{radoperasjoner}, og målet er å få en matrise som er på
\emph{trappeform}.  Denne prosessen kalles \emph{gausseliminasjon}.


\section*{Radoperasjoner}

Følgende tre måter å endre en matrise på kalles
\defterm{radoperasjoner}:
\begin{enumerate}
\item Gange alle tallene i en rad med det samme tallet (ikke~$0$).
\item Legge til (et multiplum av) en rad i en annen.
\item Bytte rekkefølge på radene.
\end{enumerate}

Vi sier at to matriser er \defterm{radekvivalente} hvis vi kan komme
fra den ene til den andre ved å utføre en eller flere radoperasjoner.
Vi bruker notasjonen $M \roweq N$ for å si at to matriser $M$ og~$N$
er radekvivalente.

\begin{ex}
\label{ex:radekvivalent}
Disse matrisene er radekvivalente, siden vi får den andre matrisen fra
den første ved å gange øverste rad med~$4$:
\begin{align*}
\begin{amatrix}{2}
 2 & 5 & 0 \\
 1 & 7 & 4
\end{amatrix}
&\roweq
\begin{amatrix}{2}
 8 & 20 & 0 \\
 1 &  7 & 4
\end{amatrix}
\end{align*}
Merk at vi også kan gå motsatt vei: Ved å gange øverste rad i den
andre matrisen med $1/4$ får vi tilbake den første matrisen.

Disse to matrisene er også radekvivalente:
\begin{align*}
\begin{amatrix}{2}
 3 & 1 & 2 \\
 9 & 5 & 7
\end{amatrix}
&\roweq
\begin{amatrix}{2}
 3 & 1 & 2 \\
 0 & 2 & 1
\end{amatrix}
\end{align*}
Her har vi brukt den andre typen radoperasjon: Vi la til $-3$ ganger
øverste rad i nederste rad for å komme fra den første matrisen til den
andre.  Merk igjen at vi også kan gå motsatt vei: Ved å legge til $3$
ganger øverste rad i nederste rad, kommer vi fra den andre matrisen
til den første.
\end{ex}

Hele poenget med radoperasjoner er at det å utføre en radoperasjon på
en totalmatrise tilsvarer å skrive om likningssystemet til et nytt
system som er ekvivalent med det opprinnelige.  Vi formulerer dette
som et teorem:

\begin{thm}
\label{thm:radekvivalens}
Hvis to likningssystemer har radekvivalente totalmatriser, så er de to
likningssystemene ekvivalente.
\end{thm}
\begin{proof}
For å bevise dette, er det nok å vise at det å gjøre en radoperasjon
på totalmatrisen til et likningssystem tilsvarer å gjøre en gyldig
omskrivning av systemet selv.

Den første typen radoperasjon -- å gange alle tallene i en rad med
samme tall -- tilsvarer å gange med det samme tallet på begge sider av
en ligning.  Litt mer detaljert: La oss si at
\[
a_{i1}\ a_{i2}\ \cdots\ a_{in}\ |\ b_i
\]
er en av radene i totalmatrisen, og at vi ganger opp denne med
tallet~$c$ slik at vi får:
\[
(c a_{i1})\ (c a_{i2})\ \cdots\ (c a_{in})\ |\ (c b_i)
\]
Dette tilsvarer at vi bytter ut likningen
\[
a_{i1} x_1 + a_{i2} x_2 + \cdots + a_{in} x_n = b_i
\]
med den nye likningen
\[
(c a_{i1}) x_1 + (c a_{i2}) x_2 + \cdots + (c a_{in}) x_n = c b_i.
\]
Men det er klart at hvis den opprinnelige likningen var sann, så må
også den nye være det.  Og siden det ikke tillates at tallet~$c$ som
vi ganger med er~$0$, så har vi også det motsatte: Hvis den nye
likningen er sann, så må også den opprinnelige være det.  Altså gjør
vi ingen endring i løsningene av likningssystemet ved å utføre denne
typen radoperasjon.

For den andre typen radoperasjon -- legge til et multiplum av en rad i
en annen -- kan vi på tilsvarende måte se at den nye raden vi lager
tilsvarer en likning som må være sann hvis de gamle likningene var
sanne.  Sett at vi legger til $c$ ganger rad~$i$ i rad~$j$.  Dette
tilsvarer at vi ganger opp den $i$-te likningen med~$c$, og legger til
resultatet i den $j$-te likningen.  Alle løsninger av de gamle
likningene må da også være løsninger av denne nye likningen.  Dessuten
kan vi komme tilbake til det gamle systemet (ved å legge til $-c$
ganger rad~$i$ i rad~$j$), og dermed må alle løsninger av det nye
systemet også være løsninger av det gamle.

Den tredje og siste typen radoperasjon -- bytte rekkefølge på radene
-- gjør åpenbart ingen endringer i løsningene av likningssystemet,
siden dette bare tilsvarer å skrive likningene i en annen rekkefølge.
\end{proof}

\begin{ex}
\label{ex:gausseliminasjon1}
Vi gjentar regningen i eksempel~\ref{ex:gausseliminasjon}, denne
gangen ved å utføre radoperasjoner på totalmatrisen til
likningssystemet:
\begin{align*}
\begin{amatrix}{3}
1 & 2 & -2 & -5 \\
1 & 5 &  9 & 33 \\
2 & 5 & -1 &  0
\end{amatrix}
&\roweq
\begin{amatrix}{3}
1 & 2 & -2 & -5 \\
0 & 3 & 11 & 38 \\
2 & 5 & -1 &  0
\end{amatrix}
\\
&\roweq
\begin{amatrix}{3}
1 & 2 & -2 & -5 \\
0 & 3 & 11 & 38 \\
0 & 1 &  3 & 10
\end{amatrix}
\\
&\roweq
\begin{amatrix}{3}
1 & 2 & -2 & -5 \\
0 & 1 &  3 & 10 \\
0 & 3 & 11 & 38
\end{amatrix}
\\
&\roweq
\begin{amatrix}{3}
1 & 2 & -2 & -5 \\
0 & 1 &  3 & 10 \\
0 & 0 &  2 &  8
\end{amatrix}
\end{align*}
Her gjorde vi følgende radoperasjoner: Legge til $-1$ ganger første
rad i andre rad, legge til $-2$ ganger første rad i tredje rad, bytte
andre og tredje rad, og legge til $-3$ ganger andre rad i tredje rad.

Den siste matrisen her er på det som kalles trappeform, og da er det
(som vi så i eksempel~\ref{ex:gausseliminasjon}) lett å finne
løsningen.  Hvis vi vil gjøre det enda lettere, kan vi fortsette med
radoperasjoner til vi oppnår det som kalles \emph{redusert
  trappeform}:
\begin{align*}
\begin{amatrix}{3}
1 & 2 & -2 & -5 \\
0 & 1 &  3 & 10 \\
0 & 0 &  2 &  8
\end{amatrix}
&\roweq
\begin{amatrix}{3}
1 & 2 & -2 & -5 \\
0 & 1 &  3 & 10 \\
0 & 0 &  1 &  4
\end{amatrix}
\\
&\roweq
\begin{amatrix}{3}
1 & 2 &  0 &  3 \\
0 & 1 &  0 & -2 \\
0 & 0 &  1 &  4
\end{amatrix}
\\
&\roweq
\begin{amatrix}{3}
1 & 0 &  0 &  7 \\
0 & 1 &  0 & -2 \\
0 & 0 &  1 &  4
\end{amatrix}
\end{align*}
Den siste totalmatrisen her svarer til følgende likningssystem:
\[
\systeme*{
x = 7,
y = -2,
z = 4
}
\]
Her har vi altså kommet helt frem til løsningen.
\end{ex}


\section*{Trappeform}

Vi vil nå gi en presis definisjon av begrepene «trappeform» og
«redusert trappeform».  Da trenger vi også et annet begrep, nemlig
«pivotelement».

% TODO: endre til «lederelement»?  ta med «pivotposisjon», «pivotkolonne»?
\begin{defn}
Det første tallet i en rad i en matrise som ikke er~$0$ kalles
\defterm{pivotelementet} for den raden.  (En rad med bare nuller har
ikke noe pivotelement.)
\end{defn}

% TODO nevne noe om at her er det matriser uten strek i?
% eller bare ta med strek i alle matriser her,
% siden vi foreløpig bare snakker om matriser som totalmatriser for likn.systemer?

\begin{ex}
\label{ex:pivotelement}
Se på følgende matrise:
\[
\begin{bmatrix}
3 & -2 & 0 & 2 \\
0 &  0  & 5 & 12 \\
1 &  8 & 3 & 7 \\
0 &  0  & 0 & 0
\end{bmatrix}
\]
Pivotelementene her er tallet~$3$ i den øverste raden, tallet~$5$ i
den andre raden og tallet~$1$ i den tredje raden.  Den siste raden
består av bare nuller, og har derfor ikke noe pivotelement.
\end{ex}

\begin{defn}
En matrise er på \defterm{trappeform} dersom hvert pivotelement er til
høyre for alle pivotelementer i tidligere rader, og eventuelle
nullrader er helt nederst.
\end{defn}

\begin{ex}
\label{ex:trappeform}
Denne matrisen er på trappeform:
\[
\begin{bmatrix}
3 & 7 & 6 \\
0 & 1 & -2 \\
0 & 0 & 2
\end{bmatrix}
\]
Pivotelementene er $3$, $1$ og~$2$, og hvert av dem er til høyre for
alle de tidligere pivotelementene.

Denne matrisen er også på trappeform:
\[
\begin{bmatrix}
-2 & 7 & 1 & 5 \\
 0 & 0 & 4 & 9 \\
 0 & 0 & 0 & 0
\end{bmatrix}
\]

Denne matrisen er ikke på trappeform fordi nullradene ikke er samlet
nederst:
\[
\begin{bmatrix}
3 & 8 & 2 & 0 \\
0 & 0 & 0 & 0 \\
0 & 2 & 1 & 4 \\
0 & 0 & 0 & 0
\end{bmatrix}
\]

Denne matrisen ser «trappete» ut, men er likevel ikke på trappeform:
\[
\begin{bmatrix}
5 & 8 & 7 & 2 \\
0 & 4 & 1 & 6 \\
0 & 9 & 2 & 0 \\
0 & 0 & 3 & 1
\end{bmatrix}
\]
Grunnen til at den ikke er på trappeform er at pivotelementet $9$ i
tredje rad ikke er til høyre for pivotelementet i andre rad, men rett
under det isteden.
\end{ex}

\begin{defn}
En matrise er på \defterm{redusert trappeform} hvis den er på
trappeform og dessuten oppfyller:
\begin{itemize}
\item Alle pivotelementene er~$1$.
\item Alle tall som står over pivotelementer er~$0$.\qedhere
\end{itemize}
\end{defn}

Den siste matrisen i eksempel~\ref{ex:gausseliminasjon1} er på
redusert trappeform, og der så vi også hva som gjør redusert
trappeform nyttig: Løsningen av systemet kan leses av direkte.

\medskip
Det å  skrive om en matrise til trappeform
ved hjelp av radoperasjoner
kalles \defterm{gausseliminasjon}, oppkalt etter den tyske
matematikeren Carl Friedrich Gauss (1777--1855).  Noen velger også å
ha et eget navn på det å komme frem til \emph{redusert} trappeform, og
kaller den prosessen for \defterm{Gauss--Jordan-eliminasjon}, oppkalt
etter Wilhelm Jordan (1842--1899).  Vi tar det ikke så nøye med den
forskjellen, og sier «gauss\-eliminasjon» uansett.

(Disse begrepene er uansett historisk sett fullstendig misvisende.
Metoden som vi kaller gauss\-eli\-minasjon var kjent i Kina for flere
tusen år siden, og Gauss -- som riktignok var et universalgeni og fant
opp mengder av flotte ting -- har ikke egentlig så mye med den å
gjøre.)


\kapittelslutt
