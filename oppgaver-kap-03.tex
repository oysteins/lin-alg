% -*- mode: LaTeX; TeX-master: "lin-alg"; -*-
\oppgaver{3}

% TODO enkle oppgaver

\begin{oppgave}
Du finner denne papirbiten som er revet ut av en bok om lineær algebra:
\begin{center}
\begin{tikzpicture}
\draw[gray!40,line width=10,xshift=3,yshift=-3] (0,0) -- (1,.4) -- (1.2,.25) -- (1.4,-.3) -- (1.9,.1) -- (2.4,0) -- (3,.3)
 -- (3.1,-.2) -- (2.9,-.5) -- (3.1,-.6) -- (2.5,-.75) -- (2.1,-.9)
 -- (2.3,-1.2) -- (1.5,-1.6)
 -- (1.25,-1.7) -- (1.2,-1.9) -- (1,-2.1) -- (1.1,-2.3) -- (.9,-2.6)
 -- (1.1,-3.2) -- (.98,-4) -- (1.15,-4.7)
 -- (1.4,-5) -- (1.2,-5.1) -- (.7,-5.3) -- (.5,-5.2) -- (.2,-5.4) -- (-.4,-5.3)
 -- (-1,-5.5) -- (-1.4,-5.4) -- (-2,-5.5) -- (-2.6,-5.6) -- (-3,-5.4) -- (-3.5,-5.15)
 -- (-3.4,-5) -- (-3.42,-4.7) -- (-3,-4) -- (-3.1,-3)
 -- (-2.9,-1.4) -- (-3.2,-1.08) -- (-3.18,-.9) -- (-3.25,-.7) -- (-3.3,-.4) -- (-3.25,-.1)
 -- (-3.4,.1) -- (-2.7,.2)
 -- (-1.9,-.1) -- (-1.4,.2) -- cycle;
\filldraw[white] (0,0) -- (1,.4) -- (1.2,.25) -- (1.4,-.3) -- (1.9,.1) -- (2.4,0) -- (3,.3)
 -- (3.1,-.2) -- (2.9,-.5) -- (3.1,-.6) -- (2.5,-.75) -- (2.1,-.9)
 -- (2.3,-1.2) -- (1.5,-1.6)
 -- (1.25,-1.7) -- (1.2,-1.9) -- (1,-2.1) -- (1.1,-2.3) -- (.9,-2.6)
 -- (1.1,-3.2) -- (.98,-4) -- (1.15,-4.7)
 -- (1.4,-5) -- (1.2,-5.1) -- (.7,-5.3) -- (.5,-5.2) -- (.2,-5.4) -- (-.4,-5.3)
 -- (-1,-5.5) -- (-1.4,-5.4) -- (-2,-5.5) -- (-2.6,-5.6) -- (-3,-5.4) -- (-3.5,-5.15)
 -- (-3.4,-5) -- (-3.42,-4.7) -- (-3,-4) -- (-3.1,-3)
 -- (-2.9,-1.4) -- (-3.2,-1.08) -- (-3.18,-.9) -- (-3.25,-.7) -- (-3.3,-.4) -- (-3.25,-.1)
 -- (-3.4,.1) -- (-2.7,.2)
 -- (-1.9,-.1) -- (-1.4,.2) -- cycle;
\draw[black!60] (0,0) -- (1,.4) -- (1.2,.25) -- (1.4,-.3) -- (1.9,.1) -- (2.4,0) -- (3,.3)
 -- (3.1,-.2) -- (2.9,-.5) -- (3.1,-.6) -- (2.5,-.75) -- (2.1,-.9)
 -- (2.3,-1.2) -- (1.5,-1.6)
 -- (1.25,-1.7) -- (1.2,-1.9) -- (1,-2.1) -- (1.1,-2.3) -- (.9,-2.6)
 -- (1.1,-3.2) -- (.98,-4) -- (1.15,-4.7)
 -- (1.4,-5) -- (1.2,-5.1) -- (.7,-5.3) -- (.5,-5.2) -- (.2,-5.4) -- (-.4,-5.3)
 -- (-1,-5.5) -- (-1.4,-5.4) -- (-2,-5.5) -- (-2.6,-5.6) -- (-3,-5.4) -- (-3.5,-5.15)
 -- (-3.4,-5) -- (-3.42,-4.7) -- (-3,-4) -- (-3.1,-3)
 -- (-2.9,-1.4) -- (-3.2,-1.08) -- (-3.18,-.9) -- (-3.25,-.7) -- (-3.3,-.4) -- (-3.25,-.1)
 -- (-3.4,.1) -- (-2.7,.2)
 -- (-1.9,-.1) -- (-1.4,.2) -- cycle;
\clip (0,0) -- (1,.4) -- (1.2,.25) -- (1.4,-.3) -- (1.9,.1) -- (2.4,0) -- (3,.3)
 -- (3.1,-.2) -- (2.9,-.5) -- (3.1,-.6) -- (2.5,-.75) -- (2.1,-.9)
 -- (2.3,-1.2) -- (1.5,-1.6)
 -- (1.25,-1.7) -- (1.2,-1.9) -- (1,-2.1) -- (1.1,-2.3) -- (.9,-2.6)
 -- (1.1,-3.2) -- (.98,-4) -- (1.15,-4.7)
 -- (1.4,-5) -- (1.2,-5.1) -- (.7,-5.3) -- (.5,-5.2) -- (.2,-5.4) -- (-.4,-5.3)
 -- (-1,-5.5) -- (-1.4,-5.4) -- (-2,-5.5) -- (-2.6,-5.6) -- (-3,-5.4) -- (-3.5,-5.15)
 -- (-3.4,-5) -- (-3.42,-4.7) -- (-3,-4) -- (-3.1,-3)
 -- (-2.9,-1.4) -- (-3.2,-1.08) -- (-3.18,-.9) -- (-3.25,-.7) -- (-3.3,-.4) -- (-3.25,-.1)
 -- (-3.4,.1) -- (-2.7,.2)
 -- (-1.9,-.1) -- (-1.4,.2) -- cycle;
\node[anchor=north,text width=18em,align=left] at (0,0) {
\fontfamily{pnc}\selectfont
\textbf{Oppgave 3.7.}
Finn ut om hvert av disse systemene har ingen, én,
eller uendelig mange løsninger:
\begin{gather*}
\systeme{
    2x +  y - 3z = 0,
    6x + 7y + 2z = 0,
    4x + 2y +  z = 0
}
\\
\systeme{
     x - 2y + 3z = 0,
    2x +  y + 2z = 0,
    3x + 4y +  z = 0
}
\end{gather*}

\textbf{Oppgave 3.8.} Løs hvert av likningssystemene
fra forrige oppgave, dersom det er mulig.
  };
\end{tikzpicture}
\end{center}
Går det an å løse oppgaven selv om alle tallene på høyre side av
likningssystemene er revet bort?
\end{oppgave}

\begin{losning}
Selv om høyresidene mangler, så kan vi forsøke å finne ut noe om
likningssystemene.  Vi antar at systemene er skrevet opp på vanlig
form, slik at det som står på høyre side bare er konstanter.  Altså
ser vi for oss at det første systemet for eksempel kan være
\[
\small
\systeme{
    2x +  y - 3z = 5,
    6x + 7y + 2z = -8,
    4x + 2y +  z = 20
}
\qquad\text{eller}\qquad
\systeme{
    2x +  y - 3z = 9,
    6x + 7y + 2z = 0,
    4x + 2y +  z = 5
}
\]
-- men ikke noe sånt som dette:
\[
\small
\systeme{
    2x +  y - 3z = 5x + 4,
    6x + 7y + 2z = y - z,
    4x + 2y +  z = x + 2y - 8
}
\]

La oss se på det første systemet først.  Den åpenbare måten å angripe
det på er å gausseliminere, og da starter vi med å sette opp
totalmatrisen til systemet:
\[
\begin{amatrix}{3}
2 & 1 & -3 & ? \\
6 & 7 & 2 & ? \\
4 & 2 & 1 & ?
\end{amatrix}
\]
Problemet er at vi ikke vet hva som skal stå i den siste kolonnen.
Men hvor stort problem er dette?  Vi kan likevel se hva som skjer med
de tre første kolonnene når vi gausseliminerer:
\[
\begin{amatrix}{3}
2 & 1 & -3 & ? \\
6 & 7 & 2 & ? \\
4 & 2 & 1 & ?
\end{amatrix}
\roweq
\begin{amatrix}{3}
2 & 1 & -3 & ? \\
0 & 4 & 11 & ? \\
0 & 0 & 7 & ?
\end{amatrix}
\]
Her har vi fått systemet på trappeform.  Vi ser at det ikke blir noen
rader med bare nuller til venstre for streken, og at det heller ikke
blir noen frie variabler.  Dermed ser vi at uansett hva som måtte stå
i den siste kolonnen, så har systemet entydig løsning.

Hva med det andre systemet?  Hvis vi gausseliminerer (med ukjent
høyreside), så får vi:
\begin{align*}
\begin{amatrix}{3}
1 & -2 & 3 & ? \\
2 & 1 & 2 & ? \\
3 & 4 & 1 & ?
\end{amatrix}
&\roweq
\begin{amatrix}{3}
1 & -2 & 3 & ? \\
0 & 5 & -4 & ? \\
0 & 10 & -8 & ?
\end{amatrix}
\\
&\roweq
\begin{amatrix}{3}
1 & -2 & 3 & ? \\
0 & 5 & -4 & ? \\
0 & 0 & 0 & ?
\end{amatrix}
\end{align*}
Oi oi oi -- en rad med bare nuller til venstre for streken!  Det
\emph{kan} bety at systemet ikke har løsning, men det kommer an på hva
som står på høyresiden.  Men hva skjer hvis det ender opp med å bli
null på høyresiden også i nederste rad?  Da har systemet løsning, og
siden det er en fri variabel (det er ikke pivotelement i tredje
kolonne, så $z$ er fri), blir det uendelig mange løsninger.

Vi kan altså konkludere med at det andre systemet definitivt ikke har
entydig løsning.  Men vi kan ikke være sikre på om det har uendelig
mange løsninger eller ingen løsning.

Det vi imidlertid kan gjøre, er å prøve å finne en sammenheng mellom
hva som står på høyresiden og hvilket av de to tilfellene (uendelig
mange eller ingen løsninger) vi får.  Så her er et par ekstraoppgaver
til deg:
\begin{enumerate}
\item Finn tre tall til høyresiden som gjør at systemet ikke har
løsning.
\item Finn tre tall til høyresiden som gjør at systemet har uendelig
mange løsninger.
\item Gi en fullstendig beskrivelse av hvilke muligheter for
høyresiden som gjør at systemet ikke har løsning.
\end{enumerate}
\end{losning}


% TODO en oppg til som over
% * 2x3-system, høyreside 0, mangler mye av venstreside
% * 


\begin{oppgave}
Kan du lage et lineært likningssystem som har entydig løsning, men som
kan gjøres om til å \ldots
\begin{punkt}
\ldots{} ha uendelig mange løsninger \ldots
\end{punkt}
\begin{punkt}
\ldots{} ikke ha noen løsning \ldots
\end{punkt}

\medskip
\noindent\ldots{} hvis vi bytter ut tallene på høyresiden?
\end{oppgave}

\begin{losning}

% TODO
\end{losning}


\begin{oppgave}
% TODO fiks
Anta at vi har et likningssystem med $m$~likninger og~$n$ ukjente.
Hvilke av de ni forskjellige tilfellene i følgende tabell er mulige?
\[
\begin{array}{r|c|c|c|}
                                & m < n & m = n & m > n \\ \hline
\text{ingen løsninger}          &       &       &       \\ \hline
\text{én løsning}               &       &       &       \\ \hline
\text{uendelig mange løsninger} &       &       &       \\ \hline
\end{array}
\]
\end{oppgave}

\begin{losning}

La 1 betegne mulig og 0 umulig:
\[
\begin{array}{r|c|c|c|}
& m < n & m = n & m > n \\ \hline
\text{ingen løsninger}          &   1   &   1   &   1   \\ \hline
\text{én løsning}               &   0   &   1   &   1   \\ \hline
\text{uendelig mange løsninger} &   1   &   1   &   1   \\ \hline
\end{array}
\]
\end{losning}


